\documentclass[12pt]{ctexart}
\usepackage[a4paper,margin=2.2cm]{geometry}
\usepackage{graphicx}
\usepackage{amsmath, amssymb}
\usepackage{xcolor}
\usepackage{hyperref}
\usepackage{enumitem}
\hypersetup{colorlinks=true,linkcolor=black,urlcolor=cyan}

\title{在大模型与优化之间架桥:LLM 驱动的海上导管架多目标优化实践}
\author{李文浩|海洋结构智能设计团队}
\date{\today}

\begin{document}
\maketitle

\begin{abstract}
\textbf{导语:}
当海上平台一步步向深远海挺进,导管架结构的设计已不再是“加厚一点钢材”那么简单。我们提出的 OJOLLM(Offshore Jacket Optimization with LLM-driven Meta-operators)框架,将大型语言模型变成会“思考”的遗传算子,让优化过程真正懂工程语义。
\end{abstract}

\section*{00 深水导管架产业背景:海基系列的启示}
南海油气开发已经全面迈向 200--350 m 的水深区,95\% 以上的海上油气仍由固定导管架平台贡献,深水化趋势让装备需求呈现阶跃式增长(付殿福等,2025\footnote{付殿福等. 中国深水导管架平台工程技术发展现状与展望. 中国海上油气, 2025.})。以“海基一号”“陆丰 12-3”“海基二号”为代表的海基系列平台,水深覆盖 241--338 m,导管架高度达到 265--339 m,总重 2.0--3.6 万吨,主腿钢材强度从 355 MPa 升至 420 MPa,桩腿数量增至 24 根。文献还强调,海洋环境、岩土工程、结构设计、建造安装、结构监测与数字孪生需要被放在同一个闭环中,超长桩的端阻影响范围 $0.12D\sim0.83D$、X 撑跨距 120 m、首阶疲劳自振周期 20 s 以上等要素都要求优化算法懂得工程语境。我们由此洞察出机会:如果遗传算子本身具备语义理解,就能主动吸收这些跨学科知识,而不是在数值空间里盲目试探。

\section*{01 需求爆发:混合变量让传统优化步履维艰}
混合变量设计是传统多目标进化算法的痛点。导管架既要在离散截面库中挑选合理的直径与壁厚,又要持续修剪连续节点坐标以确保几何协同;危险工况下的轴力与弯矩约束高度耦合,随机扰动往往只是在噪声里打转。更糟糕的是,设计—仿真—再设计的长链路让团队把大量算力花在“验证错误方案”上,工程师不得不反复猜测哪里需要加厚、哪个角度应该微调,整个优化过程缺乏可解释性。

\section*{02 我们的思路:让 LLM 成为统一的“语义遗传算子”}
OJOLLM 的核心是把大模型嵌入遗传算法的心脏位置。我们直接将 SACS 输入块喂给模型,保留每根构件的拓扑角色和载荷语境,让它明白“这是一根主腿”“那是井口区井架”。在这一语义表示之上,模型结合仿真返回的性能数据,自主判断应优先调整截面参数还是几何坐标,并输出满足构造规则的新个体。为了让模型越用越聪明,我们设计了“经验回路”:周期性抽取 Pareto 前沿与劣解,要求 LLM 总结成功策略与踩坑警示,再把这些洞察前置到后续提示词中,实现一次次在线知识蒸馏。有限元求解会输出每根杆件的 UC(利用系数)分布,如图 \ref{fig:fe} 所示,我们据此推导轴力、弯矩、屈曲等多重安全性指标,并把这些带物理意义的反馈信号回灌给 LLM,使其修改更有理有据。

\begin{figure}[h]
  \centering
  \includegraphics[width=\linewidth]{fe-example.jpg}
  \caption{有限元 UC 值计算示例:基于杆件利用系数(UC)分布推导轴力、弯矩与屈曲安全性,为 LLM 提供可解释的反馈信号}
  \label{fig:fe}
\end{figure}

\section*{03 实验案例:某 4 腿导管架多目标优化}
我们以一座典型的 4 腿导管架为例,目标是在满足关键构件轴力与弯矩利用系数约束的前提下,把结构总重压到最低。对照对象包括传统 GA、随机搜索(RS)、MOEA/D 以及不含经验回路的 LLM-GA 变体。图 \ref{fig:haiji2} 展示了自主设计建造的亚洲第一座深水导管架“海基二号”装船瞬间,它提醒我们真实工程中存在的尺度、载荷与装船窗口约束,也让模型生成的方案必须兼顾制造与安装可行性。OJOLLM 的表现主要体现在三个方面:超体积指标提升 \textbf{XX\%}(待替换为真实数据),收敛代数缩短 \textbf{YY\%} 并减少 \textbf{ZZ\%} 的仿真评估次数,最终方案的总重下降约 \textbf{AA 吨},腿部弯矩裕度提升 \textbf{BB\%}。这些数值背后意味着,大模型生成的候选更接近工程师手绘的方案草图,仿真资源得以集中在真正有潜力的结构上。

\begin{figure}[h]
  \centering
  \includegraphics[width=\linewidth]{haiji2.jpg}
  \caption{亚洲第一深水导管架“海基二号”装船现场,体现真实项目中的尺度与装船窗口约束}
  \label{fig:haiji2}
\end{figure}

\begin{figure}[h]
  \centering
  \includegraphics[width=0.85\linewidth]{jacket.drawio.png}
  \caption{OJOLLM 生成的代表性导管架拓扑(请替换为实际插图)}
\end{figure}

\section*{04 为什么有效?}
在 OJOLLM 中,LLM 拥有“路径—构件”上下文,可以同时操控离散与连续变量,避免了传统算法那种“先随机碰撞再筛选”的低效流程。经验回路相当于不断给模型补课,它会逐渐记住“哪些腿值得加厚”“哪些角度调整后更稳”,因此新一代候选天然符合工程语义,仿真调用次数也显著下降。

\section*{05 对行业的启发}
这套方法让工程知识显式化:设计规范、经验准则乃至失败案例都可以写进提示词,而不再被埋在调参经验里。语义遗传算子具备明显的迁移潜力,可平移到海上风机塔架、深水桩腿等混合变量结构;工程师也能通过修改提示词快速和算法共创,摆脱“黑箱式遗传操作”,真正形成可解释的人机协同。

\section*{06 下一步}
我们计划引入有限元云图、结构监测时程等多模态提示,让模型读懂更丰富的失效特征;探索与强化学习结合,使经验回路具备策略优化能力;同时搭建可复现的开源基准,推动“懂工程”的大模型优化范式在行业落地。

\section*{结语}
当大模型从“答题者”变成“遗传算子”,我们终于看到了语义与数值优化融合的雏形。OJOLLM 只是一个起点,它让导管架设计告别盲目试错,也为复杂工程系统打开了通向智能优化的新航道。
\end{document}
